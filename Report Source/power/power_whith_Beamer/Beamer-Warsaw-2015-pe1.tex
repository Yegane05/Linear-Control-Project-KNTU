\documentclass[10pt,xcolor=dvipsnames]{beamer}

\usepackage{amsmath,amssymb,amsfonts}
\usepackage{tikz}
\usepackage{graphicx}
\usepackage{listings}
\usepackage{booktabs}


\usetheme{Madrid}
\usecolortheme[named=blue]{structure}


\setbeamercovered{transparent}
\setbeamertemplate{navigation symbols}{}
\setbeamertemplate{headline}{}


\graphicspath{{./}{./img/}}
\setbeamertemplate{frametitle}{
	\nointerlineskip
	\begin{beamercolorbox}[wd=\paperwidth,ht=3ex,dp=1.5ex,leftskip=0.5cm,rightskip=0.5cm]{frametitle}
		\usebeamerfont{frametitle}\strut\insertframetitle\strut
	\end{beamercolorbox}
}

\title{پروژه پایانی درس کنترل خطی}
\subtitle{طراحی کنترل‌کننده سطح مایع و تحلیل دینامیکی}
\author{رضوان عسکری و یگانه طهماسبی}
\institute{دانشگاه صنعتی خواجه نصیرالدین طوسی \\ استاد راهنما: دکتر تقی‌راد}
\date{بهمن ۱۴۰۴}


\usepackage{xepersian}


\settextfont[Path=./, Scale=1.2]{Yas.ttf}
\setlatintextfont[Scale=1.0]{Times New Roman}

\defpersianfont\nas[Path=./, Scale=1.5]{IranNastaliq.ttf}

\begin{document}
	
	\begin{frame}[plain,noframenumbering]
		\centering
		\includegraphics[width=\paperwidth,height=\paperheight,keepaspectratio]{besm.jpg}
	\end{frame}
	
	\begin{frame}
		\titlepage
		\centering
		\includegraphics[width=2cm]{img/Logo.png}
	\end{frame}
	
	\begin{frame}{فهرست مطالب}
		\tableofcontents
	\end{frame}
	
	\section{معرفی پروژه و مدل‌سازی}
	
	\begin{frame}{نمای کلی سیستم}
		\begin{columns}
			\begin{column}{0.5\textwidth}
				\begin{itemize}
					\item \textbf{هدف:} کنترل ارتفاع سطح مایع
					\item \textbf{ورودی:} دبی پمپ (شیر ورودی)
					\item \textbf{خروجی:} ارتفاع آب مخزن
					\item \textbf{چالش‌ها:} رفتار غیرخطی و اشباع شیر
				\end{itemize}
			\end{column}
			\begin{column}{0.5\textwidth}
				\begin{figure}
					\centering
					\includegraphics[width=\textwidth]{img/sima.png}
					\caption{شماتیک سیستم در سیمولینک}
				\end{figure}
			\end{column}
		\end{columns}
	\end{frame}
	
	\begin{frame}{استخراج پارامترهای فیزیکی}
		پارامترهای استخراج شده از فایل \lr{InitFcn}:
		\begin{figure}
			\centering
			\includegraphics[width=0.5\textwidth]{img/PARA.png}
			\caption{پارامترهای سطح مقطع و شیرها}
		\end{figure}
		معادله دینامیکی حاکم:
		\begin{equation}
			A \frac{dh}{dt} = q_{in}(t) - C_d \sqrt{h(t)}
		\end{equation}
	\end{frame}
	
	\begin{frame}{اعتبار سنجی مدل خطی}
		مقایسه مدل خطی استخراج شده با سیستم واقعی:
		\begin{figure}
			\centering
			\includegraphics[width=0.8\textwidth]{img/4.png}
			\caption{تطابق ۶۹ درصدی به دلیل ماهیت غیرخطی}
		\end{figure}
	\end{frame}
	
	\section{تحلیل رفتار سیستم (حلقه باز)}
	
	\begin{frame}{رفتار در دامنه‌های بزرگ (غیرخطی)}
		\begin{figure}
			\centering
			\includegraphics[width=0.5\textwidth]{img/13.png}
			\caption{انحراف مدل خطی در پله‌های بزرگ}
		\end{figure}
		مدل خطی تنها در اطراف نقطه کار معتبر است.
	\end{frame}
	
	\begin{frame}{پاسخ‌های استاندارد}
		\begin{columns}
			\begin{column}{0.5\textwidth}
				\begin{figure}
					\centering
					\includegraphics[width=\textwidth]{img/15.png}
					\caption{پاسخ ضربه}
				\end{figure}
			\end{column}
			\begin{column}{0.5\textwidth}
				\begin{figure}
					\centering
					\includegraphics[width=\textwidth]{img/16.png}
					\caption{پاسخ شیب}
				\end{figure}
			\end{column}
		\end{columns}
	\end{frame}
	
	\section{تحلیل در حوزه فرکانس}
	
	\begin{frame}{نمودار بُد (Bode)}
		\begin{figure}
			\centering
			\includegraphics[width=0.6\textwidth]{img/20.png}
			\caption{رفتار فیلتر پایین‌گذر مرتبه اول}
		\end{figure}
	\end{frame}
	
	\begin{frame}{نمودار نایکوئیست (Nyquist)}
		\begin{figure}
			\centering
			\includegraphics[width=0.6\textwidth]{img/22.png}
			\caption{سیستم مینیمم فاز و پایدار است}
		\end{figure}
	\end{frame}
	
	\section{طراحی کنترل‌کننده PID}
	
	\begin{frame}{ساختار کنترلی}
		\begin{figure}
			\centering
			\includegraphics[width=1.0\textwidth]{img/PID1.png}
			\caption{کنترل‌کننده PID با فیلتر مشتق‌گیر و اشباع}
		\end{figure}
	\end{frame}
	
	\begin{frame}{روند طراحی P و PI}
		\begin{columns}
			\begin{column}{0.5\textwidth}
				\begin{figure}
					\centering
					\includegraphics[width=\textwidth]{img/p_resp.png}
					\caption{کنترل‌کننده P (دارای خطا)}
				\end{figure}
			\end{column}
			\begin{column}{0.5\textwidth}
				\begin{figure}
					\centering
					\includegraphics[width=\textwidth]{img/pi_resp.png}
					\caption{کنترل‌کننده PI (حذف خطا)}
				\end{figure}
			\end{column}
		\end{columns}
	\end{frame}
	
	\begin{frame}{نتایج نهایی PID}
		\begin{figure}
			\centering
			\includegraphics[width=0.7\textwidth]{img/finalresp.png}
			\caption{پاسخ نهایی با ضرایب بهینه}
		\end{figure}
	\end{frame}
	
	\begin{frame}{تحلیل چالش زمان نشست}
		\begin{alertblock}{چرا زمان نشست ۸ ثانیه است؟}
			رسیدن به زمان ۵ ثانیه با فیزیک این مخزن \textbf{غیرممکن} است.
		\end{alertblock}
		
		\textbf{اثبات:} حجم آب لازم برای پر کردن تانک:
		\begin{equation}
			\Delta V = A \times \Delta h = 100 \times 8 = 800 \, m^3
		\end{equation}
		حداکثر توان پمپ (شیر ۱۰۰٪ باز): $Q_{max} = 100 \, m^3/s$
		\begin{equation}
			t_{min} = \frac{\Delta V}{Q_{max}} = \frac{800}{100} = \textbf{8 seconds}
		\end{equation}
		سیستم در حالت اشباع عملگر کار می‌کند.
	\end{frame}
	
	\section{تحلیل نویز و اغتشاش}
	
	\begin{frame}{اثر فیلتر مشتق‌گیر ($N$)}
		\begin{columns}
			\begin{column}{0.5\textwidth}
				\begin{figure}
					\centering
					\includegraphics[width=\textwidth]{img/valve_bad.png}
					\caption{لرزش شیر با $N=100$}
				\end{figure}
			\end{column}
			\begin{column}{0.5\textwidth}
				\begin{figure}
					\centering
					\includegraphics[width=\textwidth]{img/valve_good.png}
					\caption{اصلاح شده با $N=1$}
				\end{figure}
			\end{column}
		\end{columns}
	\end{frame}
	

\begin{frame}{بررسی عملکرد در برابر اغتشاش (Disturbance)}
	\begin{block}{سناریوی تست}
		به منظور سنجش مقاومت سیستم در برابر عواملی مثل نشتی مخزن، یک اغتشاش پله‌ای با دامنه $-10$ در ثانیه ۳۰ به ورودی کنترلر اضافه شد.
	\end{block}
	
	\begin{figure}
		\centering
		\includegraphics[width=0.9\textwidth]{img/dist_sim.png}
		\caption{نحوه اعمال اغتشاش به ورودی سیستم در محیط سیمولینک}
	\end{figure}
\end{frame}


\begin{frame}{تحلیل پاسخ زمانی به اغتشاش}
	\begin{figure}
		\centering
		\includegraphics[width=0.5\textwidth]{img/dist_resp.png}
		\caption{بازگشت سریع سیستم به نقطه کار پس از اغتشاش}
	\end{figure}
	
	\textbf{تحلیل:}
	همانطور که مشاهده می‌شود، در لحظه $t=30$ سطح آب افت می‌کند، اما کنترل‌کننده بلافاصله با افزایش فرمان شیر، سطح آب را مجدداً به ۱۰ متر باز می‌گرداند(Robustness عالی).
\end{frame}
	
	\begin{frame}{پایداری در حضور نویز}
		\begin{figure}
			\centering
			\includegraphics[width=0.7\textwidth]{img/noise_resp.png}
			\caption{حفظ پایداری سیستم با وجود نویز سنسور}
		\end{figure}
	\end{frame}
	
	\section{نتیجه‌گیری}
	
	\begin{frame}{جمع‌بندی و پیشنهادات}
		\begin{block}{نتایج کلیدی}
			\begin{itemize}
				\item حذف کامل خطای حالت ماندگار
				\item پایداری کامل در حضور نویز و اغتشاش
				\item اثبات محدودیت فیزیکی زمان نشست
			\end{itemize}
		\end{block}
		
		\textbf{پیشنهاد برای کاهش زمان نشست به ۱ ثانیه:}
		\begin{enumerate}
			\item کاهش سطح مقطع تانک ($A$) به نصف
			\item استفاده از پمپ با دبی دو برابر ($200 \, m^3/s$)
		\end{enumerate}
	\end{frame}
	
	\section{منابع}
	\begin{frame}{منابع و مراجع}
		\begin{latin}
			\begin{thebibliography}{9}
				
				\bibitem{ogata}
				Katsuhiko Ogata.
				\newblock \emph{Modern Control Engineering}.
				\newblock 5th Edition, Prentice Hall, 2010.
				
				\bibitem{dorf}
				Richard C. Dorf and Robert H. Bishop.
				\newblock \emph{Modern Control Systems}.
				\newblock 13th Edition, Pearson, 2016.
				
				\bibitem{matlab}
				The MathWorks Inc.
				\newblock \emph{MATLAB \& Simulink Documentation: Water Tank System Modeling}.
				\newblock Natick, Massachusetts, 2024.
				
				\bibitem{taghirad}
				Hamid D. Taghirad.
				\newblock \emph{Linear Control Systems Lecture Notes}.
				\newblock K.N. Toosi University of Technology, Faculty of Electrical Engineering, Fall 2025.
				
			\end{thebibliography}
		\end{latin}
	\end{frame}
	
\begin{frame}[plain]
	\centering
	\Huge \textbf{با تشکر از توجه شما}
	
	\vspace{1cm}
	
	\includegraphics[width=3cm]{img/qrcode.png}
	
	\vspace{0.2cm}
	
	\small \textbf{اسکن جهت مشاهده کدها و مستندات}
	
	\vspace{0.2cm}
	
	\href{https://github.com/Yegane05/Linear-Control-Project-KNTU}{\lr{\texttt{github.com/Yegane05/Linear-Control-Project-KNTU}}}
	
\end{frame}

\end{document}